\documentclass{article}

\usepackage{Sweave}
\begin{document}
\input{probando-concordance}

\begin{Schunk}
\begin{Sinput}
> #Trabajando con funciones
> #function(NUMERO DE PARAMETROS){TODOS LOS DATOS QUE QUERRAMOS}
> 
> #Creando primera funcion
> suma <-function(x,y){
+   x+y
+ }
> suma(2,5)
\end{Sinput}
\begin{Soutput}
[1] 7
\end{Soutput}
\begin{Sinput}
> suma(5,2)
\end{Sinput}
\begin{Soutput}
[1] 7
\end{Soutput}
\begin{Sinput}
> a <-c(2, 5, -3, 10, 4, 9, 19)
> b <-c(-2, -1, 11, 5, 2, 7, 4)
> c <-c(1, 5)
> suma(a,b)
\end{Sinput}
\begin{Soutput}
[1]  0  4  8 15  6 16 23
\end{Soutput}
\begin{Sinput}
> suma(a,c)
\end{Sinput}
\begin{Soutput}
[1]  3 10 -2 15  5 14 20
\end{Soutput}
\end{Schunk}



\end{document}
